\documentclass[letterpaper,12pt]{article}
\usepackage{amssymb}
\usepackage{amsmath}

\setlength{\headheight}{0pt}
\setlength{\headsep}{0pt}
\setlength{\textheight}{\paperheight}
\addtolength{\textheight}{-\headheight}
\addtolength{\textheight}{-\headsep}
\addtolength{\textheight}{-\footskip}
\addtolength{\textheight}{-2in}
\setlength{\topmargin}{1in}
\addtolength{\topmargin}{-1in}

\renewcommand{\implies}{\rightarrow}
\newcommand{\pr}{\text{pr}}
\newcommand{\Prob}{\text{Pr}}
%\newcommand{\PR}{\text{PR}}
\newcommand{\dom}{\mathrm{dom\ }}
\newcommand{\rng}{\mathrm{rng\ }}


\newcommand{\A}{\mathfrak{A}}
\newcommand{\N}{\mathbb{N}}

\renewcommand{\O}{\mathfrak{O}}
\newcommand{\T}{\mathfrak{T}}
\newcommand{\proves}{\vdash}

\renewcommand{\phi}{\varphi}

\newtheorem{definition}{Definition}
\newtheorem{claim}{Claim}

\begin{document}
\section{Propositional logic}
These are the operators of \emph{classical propositional logic}:
\begin{enumerate}
\item
$\land$ means ``and'' (e.g., $a \land b$ is ``a and b are both true'')
\item
$\lor$ means ``or'' ($a \lor b$ is ``a or b is true'', i.e., ``at least one of a or b is true'')
\item
$\neg$ means ``not'' ($\neg a$ is ``a is not true'')
\item
$p \implies q$ means ``if p, then q'', equivalently, ``either p is false or q is true'', equivalently $\neg p \lor q$
\item
$p \leftrightarrow q$ means ``p if and only if q'', equivalently, both $p \implies q$ and $q \implies p$, equivalently ``p and q have the same truth value''.
\end{enumerate}

Exercises:
\begin{enumerate}
\item
Assume the sky is blue and the moon is made of rocks. Is the sentence ``if the sky is green, then the moon is made of blue cheese'' true or false?
\item
How about ``if the sky is green, then the moon is made of rocks''?
\item
How about ``the sky is green if and only if the moon is made of rocks''?
\item
If $p \implies q$ is true, is $q \implies p$ (the \emph{converse}) always true?
\item
If $p \implies q$ is true, is $\neg q \implies \neg p$ (the \emph{contrapositive}) always true?
\item
Express $a \lor b$ using only $\land$ and $\neg$. (This is called \emph{De Morgan's law}. It shows that $\land$ and $\neg$ are sufficient to express all of propositional logic.)
\item
The operator $\uparrow$ has these semantics: $a \uparrow b$ means ``either a is false or b is false''. Express all the other operators using $\uparrow$. (This operator is called \emph{NAND} or the \emph{Sheffer stroke}. By the previous exercises and definitions, it is sufficient to express just $\neg$ and $\lor$ in terms of $\uparrow$. Do you see why?)
\end{enumerate}

\newpage

\section{First-order logic}
Read $\forall$ as ``for every'' and $\exists$ as ``there exists'.

Example: let $H(x, y)$ denote ``x cuts y's hair''. Then we can write ``there is someone who cuts everyone's hair`` as:
$$\exists x \forall y H(x, y)$$
and ``everyone has someone who cuts their hair'' as:
$$\forall y \exists x H(x, y)$$

These sentences are the same, except the order of the two quantifiers has been interchanged.

Notice that $\exists x \phi(x)$ is equivalent to $\neg \forall x \neg \phi(x)$. (This relies on the assumption that at least one thing exists.)

Exercises:
\begin{enumerate}
\item
Which one of these sentences implies the other? Are they equivalent?
\item
Write the negations of both sentences.
\item
If you are familiar with the definition of a limit in calculus, write the definition of a limit using these symbols.
\end{enumerate}

\section{Basic proofs}
Let $\mathbb{N}$ denote the natural numbers, $\mathbb{Q}$ the rational numbers, and $a \in A$ the relationship ``$a$ is an element of $A$''. Prove or disprove:

\begin{enumerate}
\item
If $x$ is irrational, then $px$ is irrational for all $p \in \mathbb{Q}$.
\item
If $x$ is irrational, then $x^2$ is irrational.
\item
$\forall n \in \mathbb{N} (5n^2 > n!)$
\item
If $a$ is a positive integer, then $a^2$ is even if and only if $a$ is even.
\end{enumerate}
\end{document}
